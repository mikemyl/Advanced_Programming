\documentclass[12pt]{article}
\usepackage[english,greek]{babel} % η πρώτη γλώσσα από δεξιά είναι και η
\usepackage[utf8x]{inputenc} % για υποστήριξη ελληνικών χαρακτήρων
 \usepackage[margin=1in]{geometry} 
\usepackage{amsmath,amsthm,amssymb,amsfonts}
 
\newcounter{problem}
\newcounter{solution}

\newcommand\Problem{%
  \stepcounter{problem}%
  \textbf{\theproblem.}~%
  \setcounter{solution}{0}%
}

\newcommand\TheSolution{%
  \textbf{}\\%
}

\newcommand\ASolution{%
  \stepcounter{solution}%
  \textbf{\thesolution:}\\%
}
\parindent 0in
\parskip 1em
\begin{document}

\title{Άσκηση 2}
\author{Μιχαήλ Μυλωνάκης}
\maketitle

\selectlanguage{greek}
\section*{Εμπειρία χρήσης της \textlatin{AspectJ} }

\Problem \textit{Πόσο εύκολη ήταν η δουλειά σας?}

\TheSolution 
Σε γενικές γραμμές, η δημιουργία \textlatin{Aspects} μέσω της \textlatin{AspectJ} ήταν αρκετά απλή. Φυσικά, βασικός παράγοντας της
ευκολίας της συγγραφής των \textlatin{Aspects} ήταν και το γεγονός ότι το \textlatin{cross-cutting functionality} που θέλαμε να προσθέσουμε ήταν απλό
(έπρεπε μόνο να συγχρονίσουμε τα \textlatin{reads} και τα \textlatin{writes} σε μία \textlatin{stack}. Δαπανώντας λοιπόν τον απαιτούμενο 
χρόνο για να εξοικειωθούμε με αυτό το νέο μοντέλο προγραμματισμού, η δουλειά μας ήταν εύκολη.

\Problem \textit{Πόσο ταίριαζε η \textlatin{AspectJ} για το συγκεκριμένο σκοπό και πώς πιστεύετε ότι
γενικεύεται η εμπειρία σας σε άλλες περιστάσεις?}

\TheSolution 
Η εισαγωγή ταυτοχρονισμού σε μία υλοποίηση στοίβας αποτελεί ένα καλό παράδειγμα  \textlatin{use case} της \textlatin{AspectJ}. Ο συγχρονισμός (και γενικότερα το 
\textlatin{transaction management} αποτελούν έννοιες που είναι υπεράνω αντικειμένων, έχουν δηλαδή \textlatin{cross-cutting} εφαρμογή, και κατ' επέκταση είναι 
προτιμότερο να υπάρχει κάπου κεντρικά η λογική (ο κώδικας) που τα επιβάλλει, παρά να υπάρχει διάσπαρτη στο σώμα των μεθόδων των αντικειμένων. Η επανάληψη του ίδιου
κώδικα σε πολλά μέρη είναι \textlatin{error-prone}. Στο συγκεκριμένο \textlatin{use case} για παράδειγμα, ο κώδικας που θα έπρεπε να προστεθεί στις συναρτήσεις
\textlatin{pop} και \textlatin{push} θα ήταν πανομοιότυπος. Γενικότερα θα ταίριαζε η χρήση του \textlatin{AspectJ} για λειτουργικότητες όπως το 
\textlatin{Logging - Tracing} - όπου θα μπορούσαμε να εμφανίζουμε μηνύματα ανάλογα με τις κλήσεις μεθόδων που λαμβάνουν χώρα, για το \textlatin{Testing} - όπου θα
μπορούσαμε να αντικαταστήσουμε κάποιες κλήσεις της \textlatin{new} με \textlatin{mocks}, ή για κάποιο απλό \textlatin{Security Policy}. 'Η ακόμη θα μπορούσαμε να 
αλλάξουμε ένα μέρος της λειτουργικότητας μιας μεθόδου βιβλιοθήκης στην οποία δεν θα είχαμε πρόσβαση στον πηγαίο κώδικα (γενικότερα όπου δεν μπορούμε ή δεν
θέλουμε να πειράξουμε κάποιο \textlatin{API}. Φυσικά η παραπάνω λίστα δεν είναι εξαντλητική.

\Problem \textit{Θα χρησιμοποιούσατε την \textlatin{AspectJ} αν ήταν διαθέσιμη ευρύτατα?}

\TheSolution
Αν η λειτουργικότητα που θα θέλαμε να επιβάλλουμε ήταν διασκορπισμένη στο σύνολο των κλάσεων του προβλήματός μας (όπως για παράδειγμα τα \textlatin{use cases} 
που αναφέρθηκαν παραπάνω), και δεδομένου ότι το \textlatin{overhead} που προκύπτει από την χρήση της είναι μάλλον μικρό, τότε ναι η \textlatin{AspectJ} θα ήταν μια καλή
λύση. Θέλει ωστόσο προσοχή ως προς την υπερβολική χρήση της, καθώς όσο πιο πολύπλοκα γίνονται τα \textlatin{Aspects}, τόσο πιο δυσανάγνωστο γίνεται το πρόγραμμά
μας, και κατ' επέκταση αντί να καταφέρουμε μια πιο συμπαγή υλοποίηση, με όλη την λογική συγκεντρωμένη σε ένα 'μέρος', και άρα ένα πιο ευανάγνωστο πρόγραμμα,
το αποτέλεσμα μπορεί να είναι το αντίθετο. Όσο λοιπόν μας βοηθάει, και κάνει το πρόγραμμά μας πιο απλό  η χρήση της είναι θεμιτή. Στο σημείο που τα \textlatin{Aspect} φορτώνονται με
πολύ - δυσνόητο κώδικα, μάλλον πιο πολύ θα μας δυσκολέψει πάρα θα βοηθήσει.

\end{document}
